\documentclass[margin,line]{res}


\oddsidemargin -.5in
\evensidemargin -.5in
\textwidth=6.0in
\itemsep=0in
\parsep=0in
% if using pdflatex:
%\setlength{\pdfpagewidth}{\paperwidth}
%\setlength{\pdfpageheight}{\paperheight} 

\newenvironment{list1}{
  \begin{list}{\ding{113}}{%
      \setlength{\itemsep}{0in}
      \setlength{\parsep}{0in} \setlength{\parskip}{0in}
      \setlength{\topsep}{0in} \setlength{\partopsep}{0in} 
      \setlength{\leftmargin}{0.17in}}}{\end{list}}
\newenvironment{list2}{
  \begin{list}{$\bullet$}{%
      \setlength{\itemsep}{0in}
      \setlength{\parsep}{0in} \setlength{\parskip}{0in}
      \setlength{\topsep}{0in} \setlength{\partopsep}{0in} 
      \setlength{\leftmargin}{0.2in}}}{\end{list}}


\begin{document}

\name{Vamshi Pasunuru\vspace*{.1in}}

\begin{resume}
\section{\sc Contact Information}
\vspace{.05in}
\begin{tabular}{@{}p{3in}p{4in}}
CSA Department,             & {\it Mobile:}  (091) 8105186170 \\            
Indian Institute of Science & {\it E-mail (Work):}  vamshi@csa.iisc.ernet.in\\       
Bangalore, 560012, India.  & {\it WWW: http://pvam.github.io} \\     
\end{tabular}


\section{\sc Research Interests}

Database Systems (Query Optimization, Data Analytics), Algorithms.

\section{\sc Education}
{\bf Indian Institute of Science (IISc)}, Bangalore, Karnataka, India\\
{\em Department of Computer Science and Automation (CSA)} 
%\vspace*{-.1in}
\begin{list1}
\item[]  M.E , July 2016 (expected)
\begin{list2}
\vspace*{.05in}
%\item Dissertation Topic:  ``Nonstationary Covariance Models for
%  Spatial Data and Regression Problems'' 
%%\item Dissertation Topic:  ``Hierarchical Models for Multiple Ratings
%%  in Performance-Based\\ \hspace*{1.23in} Student Assessments.'' 
%
\item Focus : Database Systems
\item Advisor:  Jayant R. Haritsa
\end{list2}
\end{list1}

{\bf Osmania University}, Hyderabad, Telangana, India.\\
%{\em Department of Mathematics and Statistics} 
\vspace*{-.1in}
\begin{list1}
\item[] B.E, Computer Science and Engineering ,  June, 2014
\end{list1}


%\section{\sc Honors and Awards} 
%National Science Foundation Graduate Research Fellowship, 1996
%
%%\vspace*{-2.5mm}
%%NSF Vertical Integration of Research and Education in Statistics and
%%Mathematical Sciences\\ (VIGRE) teaching fellowship.
%%
%\vspace*{-2.5mm}
%Carleton College: graduated Magna Cum Laude, Honors in Biology, Phi Beta Kappa, 1993

\section{\sc Academic Experience}
{\em Teaching Assistant} \hfill {\bf August, 2015  - present}\\
Teaching Assistant in CSA, IISc for E0261: Database Management System Course offered by Prof. Jayant Haritsa for Aug2015 - till date.

\section{\sc Selected Projects}
{\bf Finding similarity across relations using Support estimation}
\hfill {\bf January-May 2015}\\
Developed a sampling based estimation algorithm to find the similarity across relations. Goal is to get the similarity by looking at fraction of data at both source and target and come up with relevant attributes to estimate similarity. 

{\bf Enhancing NECTAR}
\hfill {\bf  January-May 2015}\\
NECTAR(Nash Equilibria CompuTAtion Resource) is a platform-independent software tool for computing Nash equilibria of strategic form games and extensive form games. As part of the project, we replaced IBM's proprietary lp solver "cplex" with "lpsolve" and also made it web-accessible to reach more people. 

{\bf }
\hfill {\bf  January-May 2015}\\
NECTAR(Nash Equilibria CompuTAtion Resource) is a platform-independent software tool for computing Nash equilibria of strategic form games and extensive form games. As part of the project, we replaced IBM's proprietary lp solver "cplex" with "lpsolve" and also made it web-accessible to reach more people. 

\section{\sc Computer Skills} 
\begin{list2}
\item Statistical Packages:  R, S-Plus, BUGS; some experience
  with SAS; extensive use of C and Fortran statistical libraries.
\item Languages:  C++, Perl, Pascal, some use of Unix shell scripts,
  MPI parallel processing library.
\item Applications: Generic Mapping Tools (GMT) - Unix mapping software, \LaTeX, common Windows
  database, spreadsheet, and presentation software
\item Algorithms: Experience programming Markov Chain Monte Carlo
  simulations of Bayesian posterior distributions
\item Operating Systems:  Unix/Linux, Windows.\\ 
\end{list2}



\end{resume}
\end{document}




